\section{Enumeration algorithms}
\subsection{Model enumeration}
A simple definition of the backbone is that it is the intersection of all models of its formula. If a literal is not part of the backbone, there must exist a model that contains the negation of that literal. Therefore if we had a way to iterate over every single model of the formula and, starting with the set of both literals for every variable and removing every literal from that set that was missing in one of these models, that set would end up being the backbone of the formula. \cite{mjl10} as well as \cite{mjl15} list an algorithm that does exactly this. 

\begin{algorithm} %ueber alle models iterieren
\caption{{\sc Enumeration-based backbone computation}}
\DontPrintSemicolon
\KwIn{A satisfiable formula $F$}
\KwOut{Backbone of $F$, $\nu_r$}
$\nu_r \gets \{ x | x \in Var(F) \} \cup \{ \neg x | x \in Var(F) \}$\;
\While{$\nu_r \neq \emptyset$}{
	$(outc,\nu) \gets SAT(F)$\;
	\If{$outc = \bot$}{
		$return \; \nu_r$
	}
	$\nu_r \gets \nu_r \cap \nu$\;
	$\omega_B \gets \bigvee_{l \in \nu}\neg l$\;
	$F \gets F \cup \omega_B$\;
}
%$assert(\nu_r = \emptyset)$\;
$\Return\; \nu_r$\;
\end{algorithm}

Here, found models are prevented from being found again by adding a blocking clause of said model and the algorithm terminates once all models are prohibited and the formula became unsatisfiable through this. Should it happen that the backbone estimate becomes empty because all variables have already been encountered with $\top$ and $\bot$ assigned to them, this algorithm can also return prematurely.

\subsection{Upper bound reduction}

Clearly, calculating every single model of a formula leaves room for optimization. Most models of a common boolean formula differ by small, independent differences that can just as well occur in other models. Therefore the intersection of only a handful of models can suffice to result in the backbone, as long as these models are chosen to be as different as possible. This was achieved in \cite{mjl15} as is described in algorithm \ref{alg:ibb}\footnote{In \cite{mjl15} this is called an "Iterative algorithm", but so are algorithms \ref{alg:iterTwo} and \ref{alg:bb} that occur in the same paper and have a different approach.}.

\begin{algorithm}
\caption{{\sc Refining algorithm with complement of backbone estimate}}
\label{alg:ibb}
\DontPrintSemicolon
\KwIn{A satisfiable formula $F$}
\KwOut{Backbone of $F$, $\nu_r$}
$(outc,\nu_r) \gets SAT(F)$\;
\While{$\nu_r \neq \emptyset$}{
	$bc \gets \bigvee _{l\in\nu_r}\neg l$\;
	$(outc,\nu) \gets SAT(F \cup \{bc\})$\;
	\If{$outc = \bot$}{
		$return\; \nu_r$\;
	}
	$\nu_r \gets \nu_r \cap \nu$\;
}
$\Return \; \nu_r$\;
\end{algorithm}

It generates an upper bound $\nu_r$ of the backbone by intersecting found models and inhibits this upper bound instead of individual models. This blocking clause is much more powerful, because it enforces not only that a new model is found, but also that this new model will reduce the upper bound estimation of the backbone in each iteration. 

This is because what remains after the intersection of a handful of models, are the assignments that were the same in all these models and from that we make a blocking clause that prohibits the next model to contain that particular combination of assignments. The next model will then have to be different from all previous models for at least one of the variables in the blocking clause to satisfy it.

Eventually $\nu_r$ will be reduced to the backbone. This can be easily recognized, because the blocking clause of the backbone or any of its subsets makes the formula unsatisfiable, except in the case that the formulas backbone would be empty. 

Note that it is not particularly important for the algorithm whether the blocking clauses remain in $F$ or get replaced by the next blocking clause, because the new blocking clause $bc_{i+1}$ always subsumes the previous one $bc_i$, meaning that every solution that is prohibited by $bc_{i+1}$ is also prohibited by $bc_i$ and $F \cup \{bc_i , bc_{i+1}\}$ is equivalent to $F \cup \{bc_{i+1}\}$ concerning the set of models.

This algorithm is implemented in the $Sat4J$ library under the designation $IBB$, except that prime implicants are used instead of models. For specifics on these, see chapter \ref{ss:primeImplicant}. 


