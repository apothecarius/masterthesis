Preferences turn out to be a viable method to calculate the backbone of boolean CNF formulae that are based in real world applications. They can be disadvantageous in certain difficult cases, but it is possible to configure them in a way that they can automatically adapt to these special formulae.

If the particular application allows and calls for it, it makes sense to make an in-depth analysis of the properties that you expect to encounter in the formulae of the application for which you want to compute the backbones. Choosing the right mechanics for the traits of your formula can save a lot of computation time and creating a specialized algorithm from scratch might give an even greater benefit.

When doing this, it is vital to keep in mind that every applied method not simply means more information and that there is no perfect algorithm to compute a backbone. Every formula is different and could very well not have the property that reacts well with your employed techniques. It might even be that the key lies in not applying them regularly but only once or even discarding methods that are considered the norm and could not possibly have any noteworthy drawback.


\iffalse
TODO prefbones ist besser für echtwelt beispiele, ohne prefs zuverlässiger bei komplizierten beispielen

kombination aus strategien auf fallbeispiel optimieren(?)

Guiding the behaviour of the SAT solver using preferences can be very beneficial in the case of real world applications and where it would be disadvantageous, the penalty can be kept under control. 

nicht alles was algorithmisch erlaubt ist, bringt automatisch performance vorteile

die aggressivste methode muss nicht zwangsläufig irgend einen benefit bringen

future work dazu machen


\fi