\section{Disambiguation}
\subsection{Terminology}
This thesis is an investigation on the calculation of backbones for boolean formulas in conjunctive normal form (or CNF in short). A CNF formula $F$ is a conjunction of a set of clauses $C(F)$, meaning that all of these clauses have to be satisfied to satisfy the formula. A clause $c$ in turn is a disjunction of a set of literals, meaning at least one of said literals must be fulfilled. A literal $l$ can be defined as the occurence of a boolean variable $v$ which may or not be negated and to fulfill such a literal, it's variable must be assigned $\bot$ for negated literals and $\top$ for those literals without negation. The same variable can occur in multiple clauses of the same formula but must have the same assignment in all occurences, either ($\bot$) or ($\top$). A complete assignment of all variables of $F$ (written as $Var(F)$) that leads to the formula being fulfilled is called a model. A formula for which no model can be found is called unsatisfiable. Any set of assignments that is sufficient to satisfy a formula is called an implicant and if has no subset that isn't itself an implicant it is called a primeimplicant. The variables that do not occur in an implicant are called optional.

%An implicant is a set of assignments of the variables in the problem $F$ that still satisfies $F$. The difference to models is, that here it is allowed to leave variables undefined. Let's imagine a formula where every clause contains the literal $a$, amongst other literals. Then $(a)$ would be a simple implicant for this formula, because assigning $a$ to $\top$ is sufficient to satisfy each clause. However there may also be a different implicant that does not contain $a$ at all, satisfying the clauses in a different way. If a variable $v$ does not occur in an implicant $I$, we say that $v$ is optional in $I$.

When we want to know whether a model $M$ satisfies a formula $F$, clause $c$ or literal $l$, we write $F\langle M\rangle \rightarrow \{\top,\bot\}$ or $c\langle M\rangle$ and $l\langle M\rangle$ respectively. The result is $\top$ if the assignment satisfies what it is applied to or $\bot$ if it doesn't.

%Although it might be confusing, for many purposes in satisfiability computation, a boolean variable can not only be assigned to the two truth values $\bot$ and $\top$, but also be unassigned, don't care, free or optional. This feature is not only important for SAT solver implementations, but also to calculate //kann man auch einfach darstellen, indem die variable nicht in der menge auftaucht.


The exact terminology can differ depending on the paper and project that you read. A formula can be called a problem and the assignment of a variable can be called it's phase. Sometimes assignment and literal are used interchangeably, as they both consist of a variable and a boolean value. Clauses can also be called constraints and sometimes sentences. A synonym for a formula, clause or literal being fulfilled is it being satisfied. Models can also be called solutions of formulas. The terminology for $\top$ or $\bot$ can be $(T,F)$,$(true,false)$ or $(1,0)$.



The backbone is a formula specific set of literals that contains all literals that occur in every model of said formula. We can also say that a variable is not part of the backbone, if neither it's positive or it's negative assignment is in the backbone. If we have an unsatisfiable formula, it's backbone can be considered undefinable, which is why this thesis concerns itself only with satisfiable CNF formulas.

For the context of CNF formulas, on which this thesis focuses, the term ``subsumption'' should to be explained. A clause $c_1$ subsumes another clause $c_2$ of the same formula, if and only if $c_1 \subseteq c_2$, in prose if all the literals that occur in $c_1$ also occur in $c_2$\footnote{
	One can filibuster whether $c_1$ would have to be a true subset of $c_2$. If a formula has two occurences of the exact same clause, then one of the occurences would be redundant, so the same rule could be applied here as well. In practice it makes more sense to filter out duplicates of clauses before running any computation on the formula. Similarly, you can safely drop all clauses where both literals of the same variable occur, as that clause would be satisfied in each and every possible model.}.
If $c_1$ subsumes $c_2$ in formula $F$ then this means that you can remove $c_2$ from $F$ because in terms of satisfying models, $F\textbackslash \{c_2\}$ is equivalent to $F$ as $\{c_1\}$ is equivalent to $\{c_1,c_2\}$. This is because in this case an assignment that satisfies $c_1$ also satisfies $c_2$ automatically. There is no possible assignment that satisfies clause $c_1$ that doesn't satisfy it's subsuming clause $c_2$.

A program that determines whether a boolean formula is satisfiable or not is called a $SAT\; solver$. Typically this is done by finding a model for said formula.