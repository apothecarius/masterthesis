\section{Implementation notes for Sat4J}




TODO mal schaun ob kann weg


It is advisable to habitually verify that a function does exactly what you expect it to do, because $Sat4J$ has a habit of sometimes only doing something very similar. For example, it contains a class for a dynamically resizing array called \code{VecInt}. This is a stack that you can push integers into. Internally it consists of an array that will double in size, when its capacity is exceeded while the actual size of the \code{VecInt} is stored explicitely. If you want to use its content in functions that expect an ordinary integer array, then you would be tempted to use its \code{toArray()} function. However this directly returns a reference to the internal array, including values that were removed with the \code{pop()} function before. Throughout the library, you often see the line \code{java.util.Arrays.copyOf(ints.toArray(), ints.size())} when such a \code{VecInt} is converted to an integer array. 

- wo kommen spezial selektionsheuristiken hin (Klassenname)

- genaue adresse der Backbone klasse

- internal/dimacs literals

- vecint toArray bug

- keine doku, beste hinweise geben unit tests über howto use

- prime implicant bug bei Prefbones

- menge an gelernten klauseln kann von der maschine abhängen, vorher nochmal mit synchronisiertem repo ausprobieren

TODO PB1x besser als IBB, unterschied ?????

habe irgendwo footnote dass beim benchmarken der erste call immer schlecht ist, wegen der hardware, der muss hierhin.

keine explizit auftauchenden unit klauseln, musste selber handeln und zum resetten rausnehmen

quote thore: ich dachte mehr daran, dass Du den relevanten Teil von SAT4J grob in seinen Dimensionen darstellst. Vielleicht mit sowas wie einem Klassendiagramm? Und dann könntest Du auf einer "hohen Flughöhe" beschreiben/zusammenfassen, was Du neu implementiert hast. Das ganze aber knapp halten. Ist nur eine Idee, um eine wesentliche Leistung von Dir zu betonen. Kein Muss.

Sehr interessant wäre sicher auch ein Abschnitt, mit Problemen die Du beim Umgang mit Sat4j erlebt hast. Aber nur falls dafür noch Zeit bleibt. 


addClause gibt nullpointer zurück wenn eine äquivalente 