\chapter{Introduction}
%Backbones can also be very useful in formal logic. 
Imagine a discussion about a certain issue where the case gets complicated by the fact that the two parties cannot even agree on factual real world data, but can at least both agree on logical relationships between certain hypothetical concepts that may or may not be true. In this case, you have the opportunity to set up a boolean formula that expresses the issue, with one all deciding boolean variable that determines which of the parties is right. If one of the parties should be lucky, it will turn out that it would actually be impossible that that formula would judge in favor of their opponent, no matter what the facts were, as long as the formula itself was correctly stated and applied. For this you would calculate the backbone of the formula and simply check whether it contained the all deciding variable with either positive or negative assignment. 


TODO introduction
wofür brauche ich backbones

wer hat sich damit beschäftigt

welche paper waren relevant

wie ist die thesis strukturiert, 

was wurde in den sechs monaten gemacht


use cases von backbones:
- formel reduktion um fixe variablen
- 

%Backbones can also help to simplify a formula. If a variable is part of the backbone it is safe to remove it from the formula. This prevents you from trying to find a solution to the problem with a condition that the   . A question where one of the two answers does not make sense does not need to be asked.


hab prefbones erfunden

haben prefbones paper rausgesucht

was kann ich mit prefbones noch so anstellen

wie sehen die praktischen nutzen der prefbones varianten aus.



introduction soll aus perspektive vor dem schreiben geschrieben werden (absichten)

extra chap mit related work
nutzen für endnutzer und wirtschaft.